\documentclass[11pt]{article}

\usepackage[french]{babel}
\usepackage[utf8]{inputenc}
\usepackage{fancyhdr}
\usepackage{lastpage}
\usepackage{graphicx}
\usepackage{amsmath}
% Pour colorer une cellule de tableau
\usepackage[table]{xcolor}

% Pour les dessins 
\usepackage{tikz}
\usetikzlibrary{arrows}
\usetikzlibrary{decorations.markings}
\tikzset{middlearrow/.style={
        decoration={markings,
            mark= at position 0.5 with {\arrow{#1}} ,
        },
        postaction={decorate}
    }
}
\usetikzlibrary{calc}

% Pour les subfig
\usepackage{caption}
\usepackage{subcaption}


% TOC
\addto\captionsfrench{% Replace "english" with the language you use
  \renewcommand{\contentsname}%
    {Table of Contents}%
}

\usepackage[colorlinks=true,linkcolor=red!30!black,citecolor=blue!50!black,urlcolor=blue!40!red,pdfborder={0 0 0}]{hyperref}

%\usepackage{graphvizzz}

%\usepackage[usenames,dvipsnames]{pstricks}
%\usepackage{epsfig}
%\usepackage{pst-grad} % For gradients
%\usepackage{pst-plot} % For axes

\usepackage{enumitem}

\usepackage{framed}

%%%%%%
% Pour mise-en-forme des fichiers Ada
%
% voir exemple en fin de ce fichier.
%
% ATTENTION, requiert encoding utf-8 (voir 2ième "\lstset" ci-dessous)
 
\usepackage{listings}
%\lstset{
%  morekeywords={abort,abs,accept,access,all,and,array,at,begin,body,
%      case,constant,declare,delay,delta,digits,do,else,elsif,end,entry,
%      exception,exit,for,function,generic,goto,if,in,is,limited,loop,
%      mod,new,not,null,of,or,others,out,package,pragma,private,
%      procedure,raise,range,record,rem,renames,return,reverse,select,
%      separate,subtype,task,terminate,then,type,use,when,while,with,
%      xor,abstract,aliased,protected,requeue,tagged,until,printf},
%  sensitive=f,
%  morecomment=[l]--,
%  morestring=[d]",
%  showstringspaces=false,
%  basicstyle=\tiny\ttfamily,
%  keywordstyle=\bf\tiny,
%  commentstyle=\itshape\tiny,
%  stringstyle=\sf\tiny,
%  extendedchars=true,
%  columns=[c]fixed
%}
\usepackage{color}
\definecolor{mygreen}{rgb}{0,0.6,0}
\definecolor{mygray}{rgb}{0.5,0.5,0.5}
\definecolor{mymauve}{rgb}{0.58,0,0.82}
\definecolor{myblue}{rgb}{0,0,0.82}
\definecolor{myred}{rgb}{1,0,0}

\lstset{%
  morekeywords={abort,abs,accept,access,all,and,array,at,begin,body,
      case,constant,declare,delay,delta,digits,do,else,elsif,end,entry,
      exception,exit,for,function,generic,goto,if,in,is,limited,loop,
      mod,new,not,null,of,or,others,out,package,pragma,private,
      procedure,raise,range,record,rem,renames,return,reverse,select,
      separate,subtype,task,terminate,then,type,use,when,while,with,
      xor,abstract,aliased,protected,requeue,tagged,until,printf},
  basicstyle=\scriptsize\ttfamily,%
  commentstyle=\color{mygreen}\footnotesize\ttfamily,%
  frameround=trBL,
  frame=single,
  breaklines=true,
  showstringspaces=false,
  numbers=left,
  numberstyle=\tiny,
  numbersep=10pt,
  language=Java,
  morekeywords={Math},
  keywordstyle=\color{myblue}\bf
}

% CI-DESSOUS: conversion des caractères accentués UTF-8 
% en caractères TeX dans les listings...
\lstset{
  literate=%
  {À}{{\`A}}1 {Â}{{\^A}}1 {Ç}{{\c{C}}}1%
  {à}{{\`a}}1 {â}{{\^a}}1 {ç}{{\c{c}}}1%
  {É}{{\'E}}1 {È}{{\`E}}1 {Ê}{{\^E}}1 {Ë}{{\"E}}1% 
  {é}{{\'e}}1 {è}{{\`e}}1 {ê}{{\^e}}1 {ë}{{\"e}}1%
  {Ï}{{\"I}}1 {Î}{{\^I}}1 {Ô}{{\^O}}1%
  {ï}{{\"i}}1 {î}{{\^i}}1 {ô}{{\^o}}1%
  {Ù}{{\`U}}1 {Û}{{\^U}}1 {Ü}{{\"U}}1%
  {ù}{{\`u}}1 {û}{{\^u}}1 {ü}{{\"u}}1%
}

%%%%%%%%%%
% TAILLE DES PAGES (A4 serré)

\setlength{\parindent}{20pt}
\setlength{\parskip}{1ex}
\setlength{\textwidth}{17cm}
%\setlength{\textwidth}{16cm}
\setlength{\textheight}{23cm}
\setlength{\oddsidemargin}{-.7cm}
\setlength{\evensidemargin}{-.7cm}
\setlength{\topmargin}{-.5in}

%%%%%%%%%%
% EN-TÊTES ET PIED DE PAGES

\pagestyle{fancyplain}
\renewcommand{\headrulewidth}{0pt}
\addtolength{\headheight}{1.6pt}
\addtolength{\headheight}{2.6pt}
\lfoot{}
\cfoot{}


%%%%%%%%%%
% titre du document

\title{Algorithms \\
	\textbf{``Hole Drilling''}}

\author{Thanh Luan, Six Cyril, Vial Loïc \\
			Rouby Thomas, Marriott Richard\\
			Poupin Pierre} 

\date{7\up{th} of November 2014}


\begin{document}

\maketitle
\tableofcontents

\section{Introduction}

\section{Blabalba}

\section{blablabllablabla}


\begin{figure}
	\centering
	\begin{subfigure}[b]{0.4\textwidth}
		\centering
		\begin{tikzpicture}
			\draw [red] (0,5) -- (5,5) -- (0,4) -- (5,4) -- (0,3) -- (5,3) -- (0,2) -- (5,2) -- (0,1) -- (5,1) -- (0,0) -- (5,0) -- cycle;
			\draw [red,middlearrow={latex}] (0,5) -- (5,5);
			\draw [red,middlearrow={latex}] (5,5) -- (0,4);
			\draw [red,middlearrow={latex}] (0,4) -- (5,4);
			\draw [red,middlearrow={latex}] (5,4) -- (0,3);
			\draw [red,middlearrow={latex}] (0,3) -- (5,3);
			\draw [red,middlearrow={latex}] (0,2) -- (5,2);
			\draw [red,middlearrow={latex}] (5,2) -- (0,1);
			\draw [red,middlearrow={latex}] (5,1) -- (0,0);
			\draw [red,middlearrow={latex}] (0,0) -- (5,0);
			\draw [red,middlearrow={latex}] (5,0) -- (0,5);
			\draw [fill] (0,0) circle [radius=0.1];
			\draw [fill] (5,0) circle [radius=0.1];
			\draw [fill] (0,1) circle [radius=0.1];
			\draw [fill] (5,1) circle [radius=0.1];
			\draw [fill] (0,2) circle [radius=0.1];
			\draw [fill] (5,2) circle [radius=0.1];
			\draw [fill] (0,3) circle [radius=0.1];
			\draw [fill] (5,3) circle [radius=0.1];
			\draw [fill] (0,4) circle [radius=0.1];
			\draw [fill] (5,4) circle [radius=0.1];
			\draw [fill] (0,5) circle [radius=0.1];
			\draw [fill] (5,5) circle [radius=0.1];
		\end{tikzpicture}
		\caption{Base algorithm}
	\end{subfigure}
	\begin{subfigure}[b]{0.4\textwidth}
		\centering
		\begin{tikzpicture}
			\draw [blue] (0,5) -- (0,0) -- (5,0) -- (5,5) -- cycle;
			\draw [blue,middlearrow={latex}] (0,5) -- (0,0);
			\draw [blue,middlearrow={latex}] (0,0) -- (5,0);
			\draw [blue,middlearrow={latex}] (5,0) -- (5,5);
			\draw [blue,middlearrow={latex}] (5,5) -- (0,5);
			\draw [fill] (0,0) circle [radius=0.1];
			\draw [fill] (5,0) circle [radius=0.1];
			\draw [fill] (0,1) circle [radius=0.1];
			\draw [fill] (5,1) circle [radius=0.1];
			\draw [fill] (0,2) circle [radius=0.1];
			\draw [fill] (5,2) circle [radius=0.1];
			\draw [fill] (0,3) circle [radius=0.1];
			\draw [fill] (5,3) circle [radius=0.1];
			\draw [fill] (0,4) circle [radius=0.1];
			\draw [fill] (5,4) circle [radius=0.1];
			\draw [fill] (0,5) circle [radius=0.1];
			\draw [fill] (5,5) circle [radius=0.1];
		\end{tikzpicture}
		\caption{Optimal solution}
	\end{subfigure}
\end{figure}



\begin{figure}[ht]
	\centering
		\begin{tikzpicture}
			\draw [red] (0,5) -- (5,5) -- (0,4) -- (5,4) -- (0,3) -- (5,3) -- (0,2) -- (5,2) -- (0,1) -- (5,1) -- (0,0) -- (5,0) ;
			\draw [red,middlearrow={latex}] (0,5) -- (5,5);
			\draw [red,middlearrow={latex}] (5,5) -- (0,4);
			\draw [red,middlearrow={latex}] (0,4) -- (5,4);
			\draw [red,middlearrow={latex}] (5,4) -- (0,3);
			\draw [red,middlearrow={latex}] (0,3) -- (5,3);
			\draw [red,middlearrow={latex}] (0,2) -- (5,2);
			\draw [red,middlearrow={latex}] (5,2) -- (0,1);
			\draw [red,middlearrow={latex}] (5,1) -- (0,0);
			\draw [red,middlearrow={latex}] (0,0) -- (5,0);
			\draw [blue] (0,5) -- (0,0) -- (5,0);
			\draw [blue,middlearrow={latex}] (0,5) -- (0,0);
			\draw [blue,middlearrow={latex}] (0,0) -- (5,0);
			\draw [blue,middlearrow={latex}] (5,0) -- (5,5);
			\draw [fill] (0,0) circle [radius=0.1];
			\draw [fill] (5,0) circle [radius=0.1];
			\draw [fill] (0,1) circle [radius=0.1];
			\draw [fill] (5,1) circle [radius=0.1];
			\draw [fill] (0,2) circle [radius=0.1];
			\draw [fill] (5,2) circle [radius=0.1];
			\draw [fill] (0,3) circle [radius=0.1];
			\draw [fill] (5,3) circle [radius=0.1];
			\draw [fill] (0,4) circle [radius=0.1];
			\draw [fill] (5,4) circle [radius=0.1];
			\draw [fill] (0,5) circle [radius=0.1];
			\draw [fill] (5,5) circle [radius=0.1];
		\end{tikzpicture}
	\caption{sdqdsf}
\end{figure}

\begin{figure}[ht]
	\def\n{3}
	\def\height{5}
	\def\width{5}
	\centering

	\begin{subfigure}[b]{0.4\textwidth}
	\centering
		\begin{tikzpicture}
			\foreach \x in {0,\width} {
				\foreach \y in {0,...,\n} {
					\draw [fill] ($(\x,\y*\height/\n)$) circle [radius=0.1];
					\draw [red,middlearrow={latex}] ($(0,\y*\height/\n)$) -- ($(\width,\y*\height/\n)$);
					\ifnum \y > 0 {
						\draw [red,middlearrow={latex}] ($(\width,\y*\height/\n)$) -- ($(0,\y*\height/\n - \height/\n)$);
					} \else
					\fi
				}
			}
			\draw [blue,middlearrow={latex}] (0,\height) -- (0,0);
			\draw [blue,middlearrow={latex}] (0,0) -- (\width,0);
			\draw [blue,middlearrow={latex}] (\width,0) -- (\width,\height);
		\end{tikzpicture}
		\caption{qsdfjn}
	\end{subfigure}

	\def\n{10}
	\begin{subfigure}[b]{0.4\textwidth}
	\centering
		\begin{tikzpicture}
			\foreach \x in {0,\width} {
				\foreach \y in {0,...,\n} {
					\draw [fill] ($(\x,\y*\height/\n)$) circle [radius=0.1];
					\draw [red,middlearrow={latex}] ($(0,\y*\height/\n)$) -- ($(\width,\y*\height/\n)$);
					\ifnum \y > 0 {
						\draw [red,middlearrow={latex}] ($(\width,\y*\height/\n)$) -- ($(0,\y*\height/\n - \height/\n)$);
					} \else
					\fi
				}
			}
			\draw [blue,middlearrow={latex}] (0,\height) -- (0,0);
			\draw [blue,middlearrow={latex}] (0,0) -- (\width,0);
			\draw [blue,middlearrow={latex}] (\width,0) -- (\width,\height);
		\end{tikzpicture}
		\caption{sdqdsf}
	\end{subfigure}
\end{figure}
\end{document}

