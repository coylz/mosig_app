\documentclass[11pt]{article}

\usepackage[french]{babel}
\usepackage[utf8]{inputenc}
\usepackage{fancyhdr}
\usepackage{lastpage}
\usepackage{graphicx}
\usepackage{amsmath}
% Pour colorer une cellule de tableau
\usepackage[table]{xcolor}

% Pour les dessins 
\usepackage{tikz}
\usetikzlibrary{arrows}
\usetikzlibrary{decorations.markings}
\tikzset{middlearrow/.style={
        decoration={markings,
            mark= at position 0.5 with {\arrow{#1}} ,
        },
        postaction={decorate}
    }
}
\usetikzlibrary{calc}
\usetikzlibrary{intersections,petri,positioning,arrows}

% Pour les subfig
\usepackage{caption}
\usepackage{subcaption}


% TOC
\addto\captionsfrench{% Replace "english" with the language you use
  \renewcommand{\contentsname}%
    {Table of Contents}%
}

\usepackage[colorlinks=true,linkcolor=blue!50!black,citecolor=blue!30!black,urlcolor=blue!40!red,pdfborder={0 0 0}]{hyperref}

%\usepackage{graphvizzz}

%\usepackage[usenames,dvipsnames]{pstricks}
%\usepackage{epsfig}
%\usepackage{pst-grad} % For gradients
%\usepackage{pst-plot} % For axes

\usepackage{enumitem}

\usepackage{framed}

%%%%%%
% Pour mise-en-forme des fichiers Ada
%
% voir exemple en fin de ce fichier.
%
% ATTENTION, requiert encoding utf-8 (voir 2ième "\lstset" ci-dessous)
 
\usepackage{listings}
%\lstset{
%  morekeywords={abort,abs,accept,access,all,and,array,at,begin,body,
%      case,constant,declare,delay,delta,digits,do,else,elsif,end,entry,
%      exception,exit,for,function,generic,goto,if,in,is,limited,loop,
%      mod,new,not,null,of,or,others,out,package,pragma,private,
%      procedure,raise,range,record,rem,renames,return,reverse,select,
%      separate,subtype,task,terminate,then,type,use,when,while,with,
%      xor,abstract,aliased,protected,requeue,tagged,until,printf},
%  sensitive=f,
%  morecomment=[l]--,
%  morestring=[d]",
%  showstringspaces=false,
%  basicstyle=\tiny\ttfamily,
%  keywordstyle=\bf\tiny,
%  commentstyle=\itshape\tiny,
%  stringstyle=\sf\tiny,
%  extendedchars=true,
%  columns=[c]fixed
%}
\usepackage{color}
\definecolor{mygreen}{rgb}{0,0.6,0}
\definecolor{mygray}{rgb}{0.5,0.5,0.5}
\definecolor{mymauve}{rgb}{0.58,0,0.82}
\definecolor{myblue}{rgb}{0,0,0.82}
\definecolor{myred}{rgb}{1,0,0}

\lstset{%
  morekeywords={abort,abs,accept,access,all,and,array,at,begin,body,
      case,constant,declare,delay,delta,digits,do,else,elsif,end,entry,
      exception,exit,for,function,generic,goto,if,in,is,limited,loop,
      mod,new,not,null,of,or,others,out,package,pragma,private,
      procedure,raise,range,record,rem,renames,return,reverse,select,
      separate,subtype,task,terminate,then,type,use,when,while,with,
      xor,abstract,aliased,protected,requeue,tagged,until,printf},
  basicstyle=\scriptsize\ttfamily,%
  commentstyle=\color{mygreen}\footnotesize\ttfamily,%
  frameround=trBL,
  frame=single,
  breaklines=true,
  showstringspaces=false,
  numbers=left,
  numberstyle=\tiny,
  numbersep=10pt,
  language=Java,
  morekeywords={Math},
  keywordstyle=\color{myblue}\bf
}

% CI-DESSOUS: conversion des caractères accentués UTF-8 
% en caractères TeX dans les listings...
\lstset{
  literate=%
  {À}{{\`A}}1 {Â}{{\^A}}1 {Ç}{{\c{C}}}1%
  {à}{{\`a}}1 {â}{{\^a}}1 {ç}{{\c{c}}}1%
  {É}{{\'E}}1 {È}{{\`E}}1 {Ê}{{\^E}}1 {Ë}{{\"E}}1% 
  {é}{{\'e}}1 {è}{{\`e}}1 {ê}{{\^e}}1 {ë}{{\"e}}1%
  {Ï}{{\"I}}1 {Î}{{\^I}}1 {Ô}{{\^O}}1%
  {ï}{{\"i}}1 {î}{{\^i}}1 {ô}{{\^o}}1%
  {Ù}{{\`U}}1 {Û}{{\^U}}1 {Ü}{{\"U}}1%
  {ù}{{\`u}}1 {û}{{\^u}}1 {ü}{{\"u}}1%
}

%%%%%%%%%%
% TAILLE DES PAGES (A4 serré)

\setlength{\parindent}{20pt}
\setlength{\parskip}{1ex}
\setlength{\textwidth}{17cm}
%\setlength{\textwidth}{16cm}
\setlength{\textheight}{23cm}
\setlength{\oddsidemargin}{-.7cm}
\setlength{\evensidemargin}{-.7cm}
\setlength{\topmargin}{-.5in}

%%%%%%%%%%
% EN-TÊTES ET PIED DE PAGES

\pagestyle{fancyplain}
\renewcommand{\headrulewidth}{0pt}
\addtolength{\headheight}{1.6pt}
\addtolength{\headheight}{2.6pt}
\lfoot{}
\cfoot{}


%%%%%%%%%%
% titre du document

\title{Algorithms \\
	\textbf{``Hole Drilling''}}

\author{Thanh Luan, Six Cyril, Vial Loïc \\
			Rouby Thomas, Poupin Pierre\\
			Marriott Richard} 

\date{7\up{th} of November 2014}


\begin{document}

\maketitle
\tableofcontents

\section{Introduction}
\subsection{The Problem}

\subsection{The assumed algorithm}



\newcommand{\notalphagraph}[1] {
	\begin{tikzpicture}[scale=0.85]
			\def\n{#1}
			\def\height{5}
			\def\width{5}
			\foreach \x in {0,\width} {
				\foreach \y in {0,...,\n} {
					\draw [fill] ($(\x,\y*\height/\n)$) circle [radius=0.1];
					\draw [red,middlearrow={latex},thick] ($(0,\y*\height/\n)$) -- ($(\width,\y*\height/\n)$);
					\ifnum \y > 0 {
						\draw [red,middlearrow={latex},thick] ($(\width,\y*\height/\n)$) -- ($(0,\y*\height/\n - \height/\n)$);
					} \else
					\fi
				}
			}
			\draw [blue,middlearrow={latex},thick] (0,\height) -- (0,0);
			\draw [blue,middlearrow={latex},thick] (0,0) -- (\width,0);
			\draw [blue,middlearrow={latex},thick] (\width,0) -- (\width,\height);
		\end{tikzpicture}
}

\newcommand{\notAlphaGraphOnlyRed}[1] {
	\begin{tikzpicture}[scale=0.85]
			\def\n{#1}
			\def\height{5}
			\def\width{5}
			\foreach \x in {0,\width} {
				\foreach \y in {0,...,\n} {
					\draw [fill] ($(\x,\y*\height/\n)$) circle [radius=0.1];
					\draw [red,middlearrow={latex},thick] ($(0,\y*\height/\n)$) -- ($(\width,\y*\height/\n)$);
					\ifnum \y > 0 {
						\draw [red,middlearrow={latex},thick] ($(\width,\y*\height/\n)$) -- ($(0,\y*\height/\n - \height/\n)$);
					} \else
					\fi
				}
			}
		\end{tikzpicture}
}

\section{Proof that the currently used algorithm is not an $\alpha$-approximation}
%%% THIS ONE IS NOT THE BEST TOP TO BOTTOM
We do not have many informations about the current algorithm. We only know that it starts from the top left corner of the board, and that
it pierces holes from top to bottom.

We assume that the algorithm makes paths that go from left to right, and then from top to bottom.

\begin{figure}[h!]
	\centering
	\notAlphaGraphOnlyRed{3}
	\caption{Path that we would obtain with the algorithm}
\end{figure}

We may now wonder whether this algorithm is an $\alpha$-approximation or not. If this was the case, we would be able to find $\alpha$ such that
\[
	\forall I \in \mathcal{I}, \; \frac{{L}_{Algorithm}}{{L}_{Optimal}} \leq \alpha
\]

Let's look at the figure \ref{fig:counter-example}. The optimal solution is drawn in \textcolor{myblue}{blue}, while the algorithm's one
is in \textcolor{myred}{red}. We can see that, in such a configuration, the number of points to drill doesn't change the length of the
optimal path. However, the more points we have, the longer the red path will be. With an infinite number of points, the red path
is theoretically infinite. Which means that it is impossible to define $\alpha$ for such an algorithm.

Hence, the algorithm is \textbf{not} an $\alpha$-approximation.


\begin{figure}
        \centering
        \begin{subfigure}[b]{0.2\textwidth}
				\notalphagraph{2}
        \end{subfigure}
		\hfill
        \begin{subfigure}[b]{0.2\textwidth}
				\notalphagraph{6}
        \end{subfigure}
		\hfill
        \begin{subfigure}[b]{0.2\textwidth}
				\notalphagraph{15}
        \end{subfigure}
        \caption{Different paths depending on the number of to-be-drilled points}
		\label{fig:counter-example}
\end{figure}


\section{A Greedy Algorithm (Pac-Man)}
\subsection{Proof that Pac-Man is not optimal}

\section{Minimal spanning tree algorithm}

\subsection{Just following the tree}
\subsubsection{Proof that this is a 2-approximation}

\subsection{Luan's algorithm}

\subsubsection{Proof that Luan's algorithm is a 2-approximation}


\subsection{}<++>


\section{Conclusion}



\begin{tikzpicture}[main node/.style={circle,fill=black!100,draw,minimum size=0.2cm,inner sep=0pt}]
	\node[main node] (first) at (0,3.5) { };
	\node[main node] (second) at (4,3.5) { };
	\node[main node] (third) at (2,0) { };
	\draw[-] (first) -- (second);
	\draw[-] (first) -- (third);
	\draw[-] (second) -- (third);
	%\node[main node] (critical) [below=of waiting] at (1,0) { };

\end{tikzpicture}
\end{document}
