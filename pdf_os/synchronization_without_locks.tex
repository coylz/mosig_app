\documentclass[a4paper,10pt]{article}

%%%% PRATIQUE POUR LES ALINEAS CHIANTS
\usepackage{indentfirst}

%%%% POUR L'OPTION LABEL= %%%
\usepackage{enumitem}

\setlength{\parindent}{30pt}
\setlength{\parskip}{1ex}
\setlength{\textwidth}{15cm}
\setlength{\textheight}{24cm}
\setlength{\oddsidemargin}{0.2cm}
\setlength{\evensidemargin}{-.7cm}
\setlength{\topmargin}{-.5in}

\usepackage{graphicx}
\usepackage{titling}
\usepackage{listings}
\lstset{%
  basicstyle=\scriptsize\sffamily,%
  commentstyle=\footnotesize\ttfamily,%
  frameround=trBL,
  frame=single,
  breaklines=true,
  showstringspaces=false,
  numbers=left,
  numberstyle=\tiny,
  numbersep=10pt,
  keywordstyle=\bf
}
\newcommand{\subtitle}[1]{%
  \posttitle{%
    \par\end{center}
    \begin{center}\large#1\end{center}
    \vskip0.5em}%
}
\title{\textbf{Synchronization without locks}}
\subtitle{M1 MoSIG : Operating Systems}
\author{Poupin Pierre \and Rouby Thomas}
\date{18/11/2014}

\begin{document}
\maketitle
%\begin{abstract}
%This document is our report of the first practical session. It contains our design choices along with the results of our implementation.	
%\end{abstract}


Synchronization primitives are costly :
\begin{itemize}
  \item spinlocks :
  \begin{itemize}
    \item atomic instruction that locks the memory bus
    \item active wait
  \end{itemize}
  \item semaphore/monitor :
  \begin{itemize}
    \item risk of being unscheduled : cost of switch
    \item entering in the kernel
  \end{itemize}
\end{itemize}

Notice that actual implementation of locks in system such as Linux is a mix between spinlock and unscheduling of the thread.

Other solutions :

\begin{itemize}
  \item algorithmic changes : no more critical sections
  
  Example : Producer/Consumer
  
  \begin{center}
size\\buffer\\head

      \begin{tabular}{c|c}
        Producer & Consumer\\
        while(size==N) {} & while(size==0) {}\\
        buffer[(head+size)\%N]= object & result = buffer[head]\\
         & head=(head+1)\%N \\        
        size++ & size -- \\
      \end{tabular}
    \end{center}
  
  
  \begin{itemize}
    \item forget about size
    \item use head and tail
    \item single producer/consumer
    \item SC
  \end{itemize}

So we get :
\begin{center}
size\\buffer\\head

      \begin{tabular}{c|c}
        Producer & Consumer\\
        while((tail+1)\%N==head) {} & while(head==tail) {}\\
        buffer[tail]= object & result = buffer[head]\\
        tail = (tail+1)\%N & head=(head+1)\%N \\        
      \end{tabular}
    \end{center}    

\item take advantage of atomic instructions

Example : 
\begin{verbatim}

Compare_and_Swap(address,old,new)
 if(*address ==old) {
    *address=new;
    return 1;
 }
 else return 0;
\end{verbatim}
can be used to write an atomic stack :

\begin{verbatim}
  void atomic_push(stack* s, int value) {
      item = malloc(sizeof(node));
      item->val = value;
      do {
          item->next = *s;
      } while (compare_and_swap(s, item->next, item);
  }
\end{verbatim}

Similar for atomic\_pop (try to set s to *s-$>$next atomically)

\end{itemize}

Other performance issues include :

\begin{itemize}
  \item locks locality: with non uniform memories or with non uniform caches. $=>$ more elaborate locking structures (hierarchical, or distributed and made coherent)
  \item bottlenecks: if a single lock is used by all the threads $=>$ it's better to separate the work into independant parts and use several locks.
\end{itemize}

\end{document}
